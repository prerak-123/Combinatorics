\documentclass[10pt]{article}
\usepackage{NotesTeX,lipsum}
\graphicspath{ {/Images/} }
\title{Combinatorics}
\emailAdd{preerakcontractor@gmail.com}
\author{Prerak Contractor}

\begin{document}
\maketitle
\pagestyle{fancynotes}
\part{Graph Theory}

\section{Isomorphism}

\begin{definition}
    Two graphs $G$ and $G'$ are isomorphic if there exists one to one correspondence from vertices in $G$ to $G'$ such that a pair of vertices are adjacent in $G$ if and only if the corresponding vertices in $G'$ are adjacent.
\end{definition}

Some common ideas to check (or disprove) if two graphs are isomorphic:

\begin{itemize}
    \item Number of vertices must be equal.
    \item Number of vertices for any given degree must be equal.
    \item Each subgraph in $G$ must have an isomorphic subgraph in $G'$.
    \item Symmetries can be exploited to find matching vertices in two graphs.
\end{itemize}

\remark{These set of conditions are not exhaustive to show if two graphs are isomorphic}

\section{Edge Counting}

\begin{theorem}
    For any graph, sum of degrees of vertices is twice the number of edges in graph.
\end{theorem}

Some applications of above theorem:

\begin{itemize}
    \item There must be even number of vertices with odd degree.
    \item If a graph has only two vertices with odd degree, there must be a path connecting those two vertices. Otherwise, they would be in different connected components, however such subgraphs cannot exist since they will contain only one (odd number) vertex of odd degree. 
\end{itemize}

\begin{problem}
    Two people start at locations $A$ and $Z$ which are at same elevation of a mountain range on opposite sides of a summit $M$. Prove that if there is no point lower than $A$ or higher than $M$, then it is always possible for them to reach summit such that they are at the same level at every point of time.
\end{problem}

\begin{solution}
    Construct a graph as follows:

    \begin{itemize}
        \item Vertices are tuples $(P_L, P_R)$, with points $P_L$ and $P_R$ being on left and right side of summit $M$ respectively and either one (or both) are local peaks or valleys.
        \item Edge exists between two vertices $(P_L, P_R)$ and $(P_L', P_{R'})$ if two people can walk from $P_L$ to $P_L'$ and $P_R$ to $P_{R'}$ by walking in same direction (either up or down).
    \end{itemize}

    The problem now reduces to showing that there exists a path from $(A, Z)$ to $(M,M)$. It can be shown that the vertices $(A, Z)$ and $(M,M)$ have degree one (one can only go in one of the two directions from these points and stop when one reaches a local peak or valley) and all other vertices have degrees 2 or 4. 

    Then there must exist a path between the two vertices since they are the only two vertices of odd degree.
\end{solution}

\section{Bipartite Graphs}

\begin{definition}
    A graph is bipartite if we can partition it into two sets such that no two vertices in same set are adjacent.
\end{definition}

\begin{theorem}
    A graph is bipartite if and only if each cycle has even length (length here means number of edges in graph).
\end{theorem}

\section{Planar Graphs}

\begin{definition}
    A graph is planar if it can be drawn on a plane with no two edge intersecting.
\end{definition}

\subsection{Circle-cord method}

A method to check if a graph is planar. Note that this method will not always disprove if a graph is planar.

First, find a cycle in graph which covers all the vertices, and represent it as a circle.

Now pick an edge and draw it as in-circle or out-circle chord (doesn't matter which). This will force some edges to be out-circle or in-circle chord. Draw these edges. Keep doing this. If we find an edge which cannot be either in-circle or out-circle, the graph is not planar. If the process terminate, we have found a representation of the graph as a planar graph.

\begin{definition}
    The graph $K_5$ is the complete graph with five vertices, that is every vertex is connected to every other vertex.

    \vspace{1em}

    The graph $K_{3,3}$ is the complete bipartite graph with 3 vertices in each partition, that is, every vertex in one partition is adjacent to every vertex in other partition.

    \vspace{1em}

    A graph $G'$ is a configuration of another graph $G$ if it is obtained by adding vertices in between edges, that is an edge $ab$ is converted to two edges $ac$ and $cb$ by introducing a new vertex $c$.
\end{definition}

\begin{theorem}
    \textbf{Kuratowski, 1930}

    A graph is planar if and only if it does not contain a subgraph that is a configuration of $K_5$ or $K_{3,3}$
\end{theorem}

\begin{definition}
    Region in a planar representation of graph is an area bounded by edges. This definition also includes areas which extend to infinity. For example, in the graph $C_3$ (cycle of length 3), there are two regions, one inside the cycle and one outside.
\end{definition}

\begin{theorem}
    \textbf{Euler's Formula (1752)}

    For a connected planar graph $G$ with $r$ regions, $v$ vertices and $e$ edges, the following holds:

    \begin{equation}
        r = e - v + 2
    \end{equation}
\end{theorem}

\begin{proof}
    We can construct an inductive proof. The theorem is true for a connected graph with two vertices and one edge. Consider it is true for all graphs having $n-1$ edges.

    Now consider adding another edge. If it connects two vertices $x$, $y$ (they must be in same region) which were in the graph, it adds new region and new edge, and hence formula is still satisfied. If a new vertex is added, the number of regions remain the same but an edge and a vertex increases, and hence formula remains valid. We cannot add two new vertices and connect them using new edge as otherwise the graph will not be connected.
\end{proof}

\begin{theorem}
    For a general connected planar graph with more than one edge, the following inequality must hold:

    \begin{equation}
        e \leq 3v - 6
    \end{equation}

    And for bipartite connected planar graph, we get a tighter constraint:

    \begin{equation}
        e \leq 2v - 4
    \end{equation}
\end{theorem}

\begin{proof}
    Consider degree of region as number of edges incident to region. An edge may be counted twice in degree if the region is on both sides of the edge.

    Each region must have degree at least 3 (since region of degree two is possible only in graph with one edge). Also, sum of degrees of regions is equal to $2e$ (each edge is counted twice, one in each of the two boundaries on the two sides of the edge.) Hence, ($d$ represented degree), $d  = 2e \geq 3r \implies r \leq 2e / 3 $. Using Euler's formula, $2e / 3 \geq r = e - v + 2 \implies e \leq 3v - 6$.

    A similar argument for bipartite graph where each cycle has even length and hence every region must have degree at least 4 gives the second inequality.
\end{proof}
\end{document}